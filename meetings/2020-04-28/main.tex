\documentclass{../presentation}

\begin{document}

\frame[plain]{\titlepage}

\begin{frame}{JSON\hyp{}Format}
    \lstinputlisting[style=context,xleftmargin=2em]{spec.json}
\end{frame}

\begin{frame}{Implementierung}
    \begin{itemize}
        \item Das Lesen und Schreiben dieses Formats ist implementiert.
        \item Parsen von JSON durch die Bibliothek \href{https://github.com/nlohmann/json}
            {nlohmann/json}, wie in der BA
        \item Tests sind vorhanden und laufen erfolgreich.
    \end{itemize}
\end{frame}

\begin{frame}{Tooling}
    \begin{itemize}
        \item C++-Standard: ISO-C++ 17
        \item Compiler-Flags: \monospaced{-Werror -Wall -Wextra -Wpedantic -O2}
        \item Build: CMake
        \item \href{https://webkit.org/code-style-guidelines/}{Webkit Styleguide}
            \begin{itemize}
                \item Umsetzung durch \monospaced{clang-format} und Git Hooks
            \end{itemize}
        \item Dokumentation: Javadoc
        \item Tests: \href{https://github.com/onqtam/doctest}{onqtam/doctest}
        \item Git:
            \begin{itemize}
                \item \href{https://zivgitlab.uni-muenster.de/b_rips01/pjs-ss20}{Repo} auf dem
                    Gitlab des ZIV
                \item Gitflow Workflow
                \item Rebasen statt Mergen
            \end{itemize}
    \end{itemize}
\end{frame}

\begin{frame}{Weiteres Vorgehen}
    Eckpunkte:
    \begin{itemize}
        \item Rendering eines \emph{einzigen} Top-Level-Würfel
        \item Beliebig tief verschachtelt
        \item Ohne Farbe und Größe
    \end{itemize}
    D.h. konkret:
    \begin{itemize}
        \item Render-Loop erstellen
        \item Für jeden zu rendernden Würfel ein Vertex-Buffer-Objekt erstellen
        \item Mit der Render-Order experimentieren
        \item Zwei Systeme:
            \begin{enumerate}
                \item Zum Iterieren durch die inneren Würfel
                \item Zum Rendern der Hierarchie
            \end{enumerate}
    \end{itemize}
\end{frame}

\end{document}

% vi: tw=100
