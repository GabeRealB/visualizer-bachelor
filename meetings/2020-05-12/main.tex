\documentclass{../presentation}

\begin{document}

\frame[plain]{\titlepage}

\begin{frame}{Derzeitiger Stand}
    \begin{itemize}
        \item OpenGL wird initialisiert
        \item Das Programm wurde in Scenen und SubScenen unterteilt
        \item Data-oriented design
    \end{itemize}
\end{frame}

\begin{frame}{Scene}
    \begin{itemize}
        \item Eine Scene besteht aus mehreren SubScenes
        \item Jede SubScene wird unabhängig von einander aktualisiert und gezeichnet
        \item Die einzelnen Bilder werden im nachhinein zusammen ausgegeben
        \item Eine SubScene besteht aus einer Camera und mehreren SceneObjects
    \end{itemize}
\end{frame}

\begin{frame}{SceneObject}
    \begin{itemize}
        \item Ein SceneObject ist eine Ansammlung von Komponenten
        \item Bisherige Komponente sind Transform und CubeTickInfo
        \item Transform: Position, Scalierung und Rotation
        \item CubeTickInfo: Daten zum aktualisieren eines Würfels
    \end{itemize}
\end{frame}

\begin{frame}{Buffer}
    \begin{itemize}
        \item GenericBuffer: Ein generischer OpenGL buffer
        \item VertexAttributeBuffer: Spezialisierung von GenericBuffer
        \item Mesh: Eine Ansammlung von buffer   
    \end{itemize}
\end{frame}

\begin{frame}{Weiteres Vorgehen}
    \begin{itemize}
        \item Laden einer Scene implementieren
        \item Das rendern einer SubScene implementieren
        \item Das rendern einer Scene implementiern
    \end{itemize}
\end{frame}

\end{document}

% vi: tw=100
