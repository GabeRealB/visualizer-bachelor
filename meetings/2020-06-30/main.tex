\documentclass{../presentation}

\begin{document}

\frame[plain]{\titlepage}

\begin{frame}{Entwurf: Programm Optionen}
    \lstinputlisting[style=context,xleftmargin=2em]{visconfig_options.json}
\end{frame}

\begin{frame}{Entwurf: Globale Resourcen}
    \lstinputlisting[style=context,xleftmargin=2em]{visconfig_assets_1.json}
\end{frame}

\begin{frame}{Entwurf: Globale Resourcen}
    \lstinputlisting[style=context,xleftmargin=2em]{visconfig_assets_2.json}
\end{frame}

\begin{frame}{Entwurf: Globale Resourcen}
    \lstinputlisting[style=context,xleftmargin=2em]{visconfig_assets_3.json}
\end{frame}

\begin{frame}{Entwurf: Äußerer Würfel}
    \begin{itemize}
        \item cube: Identifikations komponente
        \item mesh: Vertices und Edges
        \item texture2D: Textur
        \item color: model.json[][ "color" ][ "tile" ]
        \item layer: wird benutzt um Kameras zu filtern
        \item transform: Position, Rotation, TILE\_SIZE
        \item iterate\_implicit: Simuliert das aktuelle Verhalten
    \end{itemize}
\end{frame}

\begin{frame}{Entwurf: Innerer Würfel}
    \begin{itemize}
        \item Wie ein äußerer Würfel
        \item parent: Vater Würfel
        \item transform: Ist relativ zum Vater Würfel
    \end{itemize}
\end{frame}

\begin{frame}{Entwurf: Kamera}
    \lstinputlisting[style=context,xleftmargin=2em]{visconfig_camera.json}
\end{frame}

\begin{frame}{Entwurf: Ausgabe}
    \lstinputlisting[style=context,xleftmargin=2em]{visconfig_composit.json}
\end{frame}

\end{document}

% vi: tw=100
